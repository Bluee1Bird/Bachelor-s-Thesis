\chapter{Background} \label{Background}


This chapter introduces the background information that is helpful to understand this Bachelor's thesis. 
First, it explains what the USB protocol is and how it works, the kind of attack this thesis deals with, and finally, the hardware that is involved, specifically the O.MG cable. 

\section{The USB Protocol} \label{TheUSBProtocol}

The Universal Serial Bus (USB) standard \cite{WaybackMachine2018} was first published in 1996. It was developed in a collaboration between the tech companies Compaq, DEC, IBM, Intel, Microsoft, NEC, and Nortel as a new standard to connect slow peripheral devices. To this end, they founded a new non-profit, called the USB Implementation Forum \cite{USBIFUSBIF}. Its goals were:
\begin{itemize}
    \item "Ease of use for PC peripheral expansion"
    \item "Low-cost solution that supports transfer rates up to 12 Mbs"
    \item "Full support for the real-time data for voice, audio, and compressed video"
    \item "Protocol flexibility for mixed-mode isochronous data transfers and asynchronous messaging"
    \item "Integration in commodity device technology"
    \item "Comprehend various PC configurations and form factors"
    \item "Provide a standard interface capable of quick diffusion into product"
    \item "Enable new classes of devices that augment the PC's capability"  
\end{itemize}
\cite[p.~23]{WaybackMachine2018}\\
The creators wanted to create a universally applicable standard protocol, for all sorts of data connections, that is also flexible, and on top of all that easy to use and low in cost. Their solution was the Universal Serial Bus.

\begin{figure}
    \centering
    \includegraphics[width=0.5\linewidth]{visuals/usbsingalgraphic.png}
    \caption{USB Cable Signals}
    \label{fig:usbsingalgraphic}
    \cite{WaybackMachine2018}
\end{figure}

Figure \ref{fig:usbsingalgraphic} depicts the construction of a USB cable. \\
The connection is built upon four signal lines; a negative (Ground) powerline GND, a positive powerline VBUS for the power supply, and two data transfer lines D+ and D-. The two physical transfer data lines are aligned as a twisted pair. This alignment protects them from outside electrical noise and does not mean that more than one logical line is available. They work together to transmit one logical signal, one single bit. USB utilizes Non-Return to Zero Inverted (NRZI) meaning that the bits on the bus are represented by a transition of physical levels, rather than their presence or absence. A power change during a time interval signals a ''1``, if the connection stays constant, it is interpreted as a ''0``. This allows a data transfer while simultaneously enabling the participating parties to synchronize their bit clocks.  \\
Since there is only one logical data line, data transfer can only happen in one direction at a time. Bi-directional communication happens in half-duplex mode, where the parties take turns sending data. To handle this restriction, a protocol manages the order of communications: the USB protocol. \\

This next section will expand on the functionalities of the USB protocol as described by \cite{nissimUSBbasedAttacks2017}. \\
A USB connection includes a host, usually a computer, and one to many peripheral devices that connect to the PC's embedded USB hub. USB devices generally fit into two categories; input/output (I/O) devices directly add capabilities to the host, while hub devices act as intermediaries to connect additional peripherals to the host. A USB cable has two main functionalities: establishing a data connection to allow communication between the connected parties, and supplying electrical power to the connected peripheral device if it is not self-powered. Such devices that draw power from a USB Host are called bus-powered. For these devices devices, it is vital to receive power before it is expected to transmit or process any data. This is why the USB connector is specifically designed with power pins that are longer than the signal pins such that the power reaches the device before any data.\\
Every USB device is equipped with a microcontroller chip that manages the USB interactions with the host. Optionally, they can feature a bootloader that permits firmware loading, for example for updates. One USB device consists of one or multiple logical subdevices that are known as device functions. For a webcam with built-in audio, this would correspond to a video and an audio function. Devices with multiple subdevices are referred to as composite devices. Each function is managed through a separate endpoint on the bus with its individual logical address. One endpoint forms one logical communication channel called a pipe, of which there are two types.  
\begin{enumerate}
    \item Message Pipe: A message pipe is used for control transfers. That means they are used for short and simple commands sent to the USB device and status responses to the host.
    \item Stream Pipe: A stream pipe is used for actual data transfer.
\end{enumerate}
Data transfer can only take place if the host directly requests it. A USB device cannot transfer information autonomously. 

To be able to communicate with any USB device, a connection must to be established and initialized. This is done with a process called 'enumeration'. It starts as soon as the physical connection is established and consists of four main steps:

\begin{enumerate}
    \item \emph{Detection}: A change in the current on the data lines is detected by the host. This means that a new USB device has been connected.
    \item \emph{Device Speed}: The speed of the device is determined by using the change on the data lines in step 1.
    \item \emph{Device Descriptors}: The USB device is reset by the host through a specific data signal. This prompts the device to send information about itself. These descriptors are used to identify the capabilities of the device.  \\
    The exchange of descriptors follows a defined order: 
    \begin{itemize}
        \item First the host will request the descriptor length and the descriptor from the device with the Get\_Device\_Descriptor command.
        \item The advice is reset again and given a unique local address by the host called via the Set\_Address command. 
        \item Lastly, the device is prompted to send its configuration by the Get\_Configuration\_Descriptor. The configuration includes a hierarchy of interface, endpoint, and class-specific descriptors.  
    \end{itemize}
    \item \emph{Loading Drivers}: Now that all the information has been exchanged the host can start using it to load the device-specific drivers that will allow control over the device. The corresponding driver is found through the USB class, the vendor ID (VID), and the product ID (PID). Most standard drivers are included in the operating system (OS) of the host. If they aren't the user has to download them manually. This concludes the enumeration process; the device is now ready for use.
\end{enumerate}




\section{The Dangers of USB} \label{TheDangersOfUSB}

It may be observed by some that the initialization process described in Section \ref{TheUSBProtocol} does not incorporate verification or authorization steps. The protocol assumes that any physically connected USB device is trustworthy and does not take any precautions to check its claims or properties. This is a big oversight and opens the door for a lot of malicious activity. 
The risks posed by this activity are exacerbated by the fact that USB devices are not perceived as a threat by a vast majority of people. Many would pick up, plug in, and even interact with USB sticks they find lying around on the ground. A study \cite{tischerUsersReallyPlug2016} conducted in 2016 found that USB sticks are a very effective attack vector. They explored whether USB sticks dropped on a university campus would be picked up and plugged in. They found that users opened one or more files on 45\% of the flash drives and that 98\% of the drives were removed from the drop location by the time the experiment had ended. Based on this, the authors estimate that between 45\% and 98\% of drives were eventually plugged in. Through a survey placed on the drives, it was concluded that the participants acted mostly out of altruistic motives although the authors speculate that many may have acted out of curiosity. The social engineering attack vector was concluded to be an ``expeditious" with a median time of connection of only 6.9 hours. \\
USB ports are ubiquitous in all kinds of environments. USB has developed into the standard for convenient power and data transfer, not only in private but also in public.  Many people do not think twice when connecting their laptop to an external display in a library, using public charging ports, or accepting a charging cable from a friend. Still, none of those interactions are protected, and every use of the USB protocol poses a risk that most are not aware of.

\begin{figure}[H]
    \centering
    \includegraphics[width=0.6\linewidth]{visuals/USB_treadmill.jpeg}
    \caption{USB is everywhere; inconspicuous USB port on a treadmill in a public gym.}
    \label{fig:USB_treadmill}
    \cite{}
\end{figure}

A myriad of different attacks are possible through such an unsecured attack vector. This paragraph describes a few of them. \\
Through USB, data can be silently downloaded without the user's permission or knowledge from computers \cite{clarkHardwareTrojanHorse2009} as well as phones \cite{SharpIdeasDownloads2006}. USB drives could be developed to be able to do data manipulation, and with the increased popularity of U3, attackers could hack U3 images and replace them for their own malicious purposes to take advantage of auto-run. 
How much damage could be done with this is best exemplified with Stuxnet \cite{kushnerRealStoryStuxnet2013}. It's a malware program that could travel via USB; a computer that came in contact with an infected USB stick would immediately be compromised. In this way, malware could spread even in an air-gaped environment. The highly sophisticated and targeted attack was discovered in 2010 and is confirmed to have damaged centrifuges that were part of Iran's Nuclear Program.\\
But an attack via USB does not have to have the dimensions of an (alleged) geopolitical intelligence mission \cite{kushnerRealStoryStuxnet2013}, there are more examples of day-to-day threats to exemplify what USB can do. On Windows XP it is possible to emulate a CD-ROM device through a USB connection and hack the U3 autoplay feature \cite{al-zarouniRealityRisksConsented2006} and data can be exfiltrated from iOS6 through a USB charging cable \cite{lauMactansInjectingMalware2013}. USB can be used to propagate attacks from one computer to another, as demonstrated in \cite{wangExploitingSmartphoneUSB2010}, or destroy a target with power surges that irreparably damage the host \cite{USBKillDevices}, data can be transferred to or from a device while charging \cite{kumarJuiceJackingUSB2020} and Fork Bomb attacks can be launched via USB \cite{efendyExploringPossibilityUSB2019}. Most importantly, however, since USB does not require authentication, it can be used to emulate other devices, namely devices that are used for human input. Human Interface Devices (HID) such as keyboards and mice present an unparalleled attack vector, where a hacker with physical access to an unlocked computer can remotely execute any action a user themselves could \cite{USBRubberDucky}, \cite{MGCable2019a}, \cite{lawalFacilitatingCyberenabledFraud2022}. \\
What such HID attacks could look like will be discussed and exemplified in this Thesis.


\subsection{Bad Hardware}

None of the attacks and only part of the defences that are discussed in this thesis would be possible without specialized hardware. This section will therefore give a brief introduction to the different microcontrollers and computers that will be mentioned in the following chapters.\\
In general, any USB device that has been modified to execute some malicious action will be referred to as BadUSB, a term coined by a BlackHat presentation by Karsten Nohl, Sascha Krissler, and Jakob Lell \cite{Srlabsbadusbblackhatv1Pdf2014}. 


\subsubsection{Arduino}

\begin{figure}[H]
    \centering
    \includegraphics[width=0.25\linewidth]{visuals/arduinomini.png}
    \caption{The Arduino pro mini, used by \cite{bojovicRisingThreatHardware2019}}
    \label{fig:ArduinoProMini}
    \cite{ArduinoProMini}
\end{figure}
Arduino \cite{ArduinoHardware} is a company that produces a range of different small microcontrollers designed to be accessible and straightforward. They are supported by the open-source Arduino platform and the Arduino IDE \cite{ArduinoArduino2024}.


\subsubsection{Teensy}

\begin{figure}[H]
    \centering
    \includegraphics[width=0.5\linewidth]{visuals/teensy.png}
    \caption{Teensy (in green) as built into Malboard}
    \label{fig:builtInTeensy}
    \cite{farhiMalboardNovelUser2019}
\end{figure}

The Teensy  \cite{TeensyUSBDevelopment} is a microcontroller system based on USB, that is, as its name suggests, tiny. It can be programmed via its USB port, is compatible with Arduino Software, and works with Mac OS, Linux, and Windows. It comes standard with solder pads. To program it, the Teensy Loader Application can be used.
A teensy can be programmed to emulate a keyboard and a mouse and change its PID and VID \cite{farhiMalboardNovelUser2019}.\\
These qualities combined with its size, make it a popular microcontroller for homemade BadUSBs. 


\subsubsection{Hak5 Hardware} \label{Hak5Hardware}

\begin{figure}[H]
    \centering
    \includegraphics[width=0.5\linewidth]{visuals/OMGCable.png}
    \caption{The O.MG cable and its Web IDE open on a phone}
    \label{fig:OMGCable}
    \cite{hak5MGCable}
\end{figure}

Hak5 is a company founded in 2005 by a group who have made it their mission to 'advance the InfoSec industry' \cite{hak5}. They have an award-winning podcast, a big YouTube channel, and a lot of penetration-testing gear. They aim to inform people of the security risks that come with tech to ultimately make the world a safer place. \\
In 2019, the O.MG cable was made available at Hak5 \cite{MGCable2019a} marking the beginning of a collaboration between the creator of the O.MG cable and founder of Mischief Gadgets, MG \cite{MGCable2019a} and the Hak5 team. They have since released many variations of the O.MG cable, including a plug, an adapter, an 'unblocker', and a cable detector \cite{hak5MischiefGadgets}. The device that will be featured in this thesis is the O.MG cable. It is plug-and-play and can therefore be used to conduct injection attacks, without having to do any hardware building yourself \cite{hak5MGCable}. 



\section{DuckyScript and the O.MG cable}


This section will give an overview of the O.MG cable, and the scripting language it is equipped with; DuckyScript. 

\subsection{The O.MG cable} \label{theOMGCable}

As described shortly in sections \ref{Hak5Hardware} and \ref{HistoryOfAttacks} the O.MG cable is a BadUSB cable invented by MG \cite{MGCable2019a}, produced by Mischief Gadgets \cite{hak5MischiefGadgets} and sold in cooperation with Hak5 on their website \cite{hak5MischiefGadgets}. \\
The cable does not have any physical markers; neither its USB ends, the cable, nor the weight gives any indication that it has so many more capabilities than normal USB cables. For the most part, it is a usual USB cable; the passive (passthrough) end does not have any special abilities, and if it is not configured to execute a boot script, the cable can be used like any other to charge an transfer data. Its malicious, active end is marked by a USB trident. This symbol is often found on USB cables to indicate their compatibility with the USB standard. The active end transmits the payloads from the device to the USB host. It is available in USB-A, USB-C, and lightning. 
To set up an O.MG cable, the first step is to flash it. This must be done before initial use or after a self-destruct sequence has been triggered. Flashing the cable requires an O.MG Programmer \cite{hak5MGCable}. The active end of the O.MG cable is plugged into the programmer, which itself is connected to a computer by a regular USB data cable. Flashing can either be done with the official open-source firmware \cite{DuckyScriptSyntaxGuide} via the setup website (https://o.mg.lol/setup/OMGCable/) or with custom firmware. After flashing the device is ready to be used. \\  
Once the active end receives power from a USB host, it establishes a short-range wireless network with the default SSID "O.MG" and password "12345678," both of which are configurable. The WebUI can be accessed through that network, on ``http://192.168.4.1.''. \\

\begin{figure}[H]
    \centering
    \includegraphics[width=1\linewidth]{visuals/O.MG_webUI.png}
    \caption{Web UI of the O.MG cable in Microsoft Edge.}
    \label{fig:O.MG_webUI}
    \cite{}
\end{figure}



As seen in Figure \ref{fig:O.MG_webUI} the WebUI offers a variety of features.

\begin{enumerate}
    \item Payloads
    \item Geofencing
    \item Self-Destruct
    \item Keylogger
    \item Keymap Viewer
    \item Partition Editor
    \item C2
    \item HIDX StealthLink
\end{enumerate}

The following subsections contain a short description of each of these features. 

\subsubsection{Payloads}

All O.MG devices that can be programmed for keyboard injection support an enhanced version of DuckyScript, as expanded upon in section \ref{DuckyScript}. \\
Payloads can be written directly in the WebUI, where they can be stored in slots. Depending on the model, the cable supports from 8-200 payload slots \cite{hak5MGCable}. All stored payloads can be loaded into the WebUI for editing and execution. The payload will execute on the device connected to the active end of the cable. Payloads can be set as boot script. This means they will be executed as soon as the active end of the cable is connected to a host. There can only be one boot script at a time. \\



\subsubsection{Geofencing}

Geofencing can be accomplished by a simple conditional statement. It allows triggering or blocking actions depending on the presence or absence of a wifi network with a specific SSID. This is accomplished by a condition that is evaluated every time the O.MG device powers up. Alternatively, the cable can wait until an SSID or BSSID is present with the 'WAIT\_FOR\_PRESENT' keyword.
It can be configured to either wait for a specific amount of time or run forever. Conversely, this feature can also be used to wait until it is out of reach of a network.

\subsubsection{Self-Destruct}

Cleanup is an important part of an attack or penetration test. This cable supports two kinds of self-destruct that can be triggered via the WebUI or with a keyword in the code.
\begin{itemize}
    \item  \textbf{Self-Destruct 1:} Completely erases all data on the cable and disconnects the data line. The cable will appear broken. To use it again, it has to be physically recovered and reflashed.  
    \item  \textbf{Self-Destruct 2:}  Erases all the data but retains the data lines, turning the cable in a normal USB cable. Reflashing is also necessary to use the cable maliciously again. 
\end{itemize}


\subsubsection{Keylogger}

When an O.MG cable connects a physical keyboard with a detachable cable to a host, it can log the keystrokes from the keyboard and display them in real time in the WebUI. In order for this to work, the keyboard has to be Full Speed USB (12mbps) and not low (1.5mbps) or high speed (480mbps). 

\subsubsection{Keymap Viewer}

Since the O.MG cable imitates keyboard presses, it is dependent on the keyboard settings of the attacked host. Input in a US keyboard will not produce the same signal to the computer as a DE\_CH keyboard even though the same key is pressed. For this reason, the language of a payload has to be set, which is explained in more detail in Section \ref{ducky\_lang}. \\
For this reason, the Keymap Viewer exists. Using keylogging the keyboard layout of a connected host can be determined. 

\subsubsection{Partition Editor}

On specific models of the O.MG cable, a user can decide how the available storage should be partitioned through the partition editor. They can decide to redirect storage resources to payload slots, individual slot size, size of storage for exfiltrated data, or keylogging storage. 

\subsubsection{C2}

A connect and control (C\&C or C2) server is useful to control devices from anywhere, not only the limited range of its network or one it is logged into. It thereby eliminates the need for physical proximity. Communication between the remote server and the device runs via HTTP and is encrypted with MonoCypher. 

\subsubsection{HIDX Stealth Link}
This feature is still in progress, but Its aim is to enable more stealthy data exfiltration. Instead of establishing a network or sending the data via the host, the cable relays the data via HID channels.  


\subsection{Ducky Script} \label{DuckyScript}

DuckyScript was invented by Darren Kitchen, the founder of Hak5. It evolved from a macro scripting language that could handle keyboard injections and delays in its first iteration to a structured and feature-rich programming language supporting injection attack-specific commands (jitter, side-channel exfiltration) as well as control flow instructions, repetitions, and functions in version 3.0 \cite{kitchenUSBRUBBERDUCKY2022}. \\
Some specific Hak5 gear like the BashBunny or LANTurtle uses an interpreted version of DuckyScript 1.0 that incorporates the BASH shell scripting language \cite{Hak5Usbrubberduckypayloads2024}. The DuckyScript version licensed for the O.MG cable is based on DuckyScript 1.0 and additionally includes some device-specific commands for the specific functionalities of the O.MG cable. The available commands include the usual keyboard keys like ENTER, GUI (Windows key), F11, PAGEUP, TAB; ESCAPE, etc. Some of the other supported commands are\cite{DuckyScriptSyntaxGuide}:
\begin{itemize}
    \item DUCKY\_LANG: set the input language of the keyboard.
    \item DELAY: Pause for a specified amount of time (in milliseconds)
    \item DEFINE. Define a variable
    \item STRING: Followed by a string that is sent as input to the host, STRINGLN follows that input with ``ENTER''.
    \item MOUSE: controls the mouse movement.
    \item USB ON/OFF: Enumerate or disconnect as USB device.
    \item JIGGLER: turn mouse jiggle on or off.
    \item REBOOT: reboot the O.MG
    \item DEFAULT\_DELAY: Puts a default delay before every command. Also available as DEFAULT\_CHAR\_DELAY to put delays in front of every STRING character.  
    \item REPEAT: Repeats the preceding value a number of times.
    \item VID/PID: sets the VID / PID.
    \item MAN / PRO / SER: set iManufacturer / iProdcut / iSerial
    \item KEYLOGGER ON / OFF: turns keylogger on or off.
    \item NOKEY: will send a null value.  
\end{itemize}

Some more options, such as SELF\_DESTRUCT were introduced in Section \ref{theOMGCable}

What a basic implementation of a Hello, World! could look like is shown in Listing \ref{lst:hello_world_duckyscript}.

\begin{lstlisting}[caption={Hello, World! in DuckyScript 1.0}, label={lst:hello_world_duckyscript}, captionpos=b]
REM Title: Hello, World!
REM Description: Write Hello, World! in the terminal
REM Target: Windows (including Powershell 2.0 or above)

DELAY 3000
GUI r
DELAY 750
STRING cmd
ENTER
DELAY 2000
STRING Hello, World!
\end{lstlisting}



\section{Chapter Conclusion}

This chapter introduced the USB standard and protocol and the vulnerabilities that come with it. It features a short history of USB and a physical and technical explanation of USB. It showed how the lack of USB authentication can be leveraged by malicious actors to target users, stealthily and efficiently. To this end, it listed a variety of different possible USB attacks and introduced the keyboard injection attack.
Lastly, a basic overview of the hardware in this thesis was given and the O.MG cable, as well as its scripting language, DuckyScript was introduced. 
Building on this background knowledge, the next chapters of this thesis elaborate on the history of HID injection attacks and eventually novel payloads and defence mechanisms. 