\chapter{Methodology and Architecture} \label{Methodology}

\section{Introduction}

This chapter will describe the point of view of an attacker during a HID injection attack. To this end it will firstly explain the evaluate existing attack payloads, describe a general architecture for an attack and lastly, it will introduce new attacks. 


\section{Ethical Considerations}
Attacks as described in this thesis can produce considerable harm to individuals, companies, and communities. For all the reasons explained and examples brought up in section \ref{TheDangersOfUSB} these attacks are not to be taken lightly and the potential for damage is real. For this exact reason it is important to raise awareness for this kind of attack, research existing vulnerabilities, new developments in the field, and how those can be counter acted. Investigating attacks in particular ones that are not available in a public GitHub repo is an important contribution to the scientific field and outweighs the negative. Just because these payloads cannot easily be found, does not mean they do not already exist. Which arguably makes them even more dangerous to the public and therefore their exploration is even more urgent. \\
In section \ref{HistoryOfDefense} this thesis explains countermeasures that can be put in place to protect oneself and in section [TODO!!] new methods will be presented, specifically against new attacks introduced in section \ref{Implementation}.
% bring up the balck/white /grey hat discussion again? or is his not scientific enough?


\section{ATT\&CK by MITRE}

ATT\&CK \cite{MITREATTCK} is an openly accessible knowledge base of adversary tactics and techniques developed by the security advisor firm MITRE \cite{WhoWeAre}. It can be used for threat modeling and to get a general overview over the different types of cyber attacks that happen in the world.
It features 14 attack categories that themselves are subdivided into 8-43 techniques. One example is the category Reconnaissance which is divided into Active Scanning, Gather Victim Host Information, Gather Victim Identity Information, Gather Victim Network Information, Gather Victim Org Information, Phishing for Information, Search Closed Sources, Search Open Technical Databases, Search Open Websites/Domains, Search Victim Owned Websites. 



\subsection{Evaluation of Existing Attack Scripts}

In the following, this paper will dive into some of those categories and techniques and evaluate whether that can be executed via keyboard injection and whether or not a script for it is available online. To this end, scripts form the official open source GitHub page for O.MG \footnote{https://github.com/hak5/omg-payloads} by Hak5 are examined. \\
It is important to note that the most basic attack that can be executed via Keyboard Injection is also the most versatile and one of the most dangerous ones. It is a simple script that downloads any malware of your choice, which as a result, could execute any attack that can be done via code.  
This analysis will therefore focus on whether an attack has been implemented solely as keyboard injection and will not feature payloads that include downloading additional malware. It will not examine all 14 categories instead highlighting the most relevant ones. 

\subsubsection{Reconnaissance}

Reconnaissance is about actively or passively gathering information, often used before an attack. The gathered information can be used to inform the planning of a bigger attack or to further additional reconnaissance efforts. This category includes scanning of network traffic, gathering host information such as name, IP, operating system, hardware information, credentials, email, or information about their network. Phishing also belongs into this category, and purchasing information about the system from legal or illegal data brokers as well as gathering publicly available information, for example from the internet. \cite{MITREATTCK}

This is a field in which keyboard injection can make considerable damage. There exist many exfiltration scripts, that download sensitive files, exfiltrate passwords, network information, and device information, or social engineer the user to enter sensitive data on malicious sites. 

For example, \verb|Harvester_OF_SORROW| \cite{OmgpayloadsPayloadsLibrary} exfiltrates login information from Firefox on Windows 10. \\ 
Other examples for password extraction are \verb|SudoSnatch| \cite{OmgpayloadsPayloadsLibrary} which exfiltrates sudo passwords, and \verb|WLAN-Windows-Passwords| \cite{OmgpayloadsPayloadsLibrary}which steals wlan passwords and sends them to the attacker via a discord webhook. 
\verb|OMGLogger| \cite{OmgpayloadsPayloadsLibrary} leverages the logging capabilities of the O.MG cable and sends the keystrokes live to the attacker's server. \\
The collection on GitHub has an entire folder for exfiltration scripts, that can find data on a network, a printer, a target's Spotify, Powershell history, log files, MySql history, FireFox browser cookies, fotos, or send periodic screenshots. \\
Similarly there is a folder with scripts on phishing  \cite{OmgpayloadsPayloadsLibrary}, however, it is less extensive. The three payloads build on the idea of faking a pop up, where they prompt the user to (re)submit login data and all require some previous installations to be present and they are all written for Linux. \\
The recon folder contains a script that does device recon using a Powershell script that can be hosted on a server and then downloaded via keyboard injection. 


\subsubsection{Resource Development}

Resource Development is what a malicious actor does when they want to establish resources that can help them mount an attack, such as getting access to specific email addresses, system accounts, or target system. It also includes setting up servers or bots that could be used for an attack or creating and cultivating accounts to build a persona.\cite{MITREATTCK}\\
Resource Development is an important part of injection attacks in conjunction with data exfiltration. This is apparent in payloads like \verb|ExfiltrateLinuxLogFiles| \cite{OmgpayloadsPayloadsLibrary}, \verb|WLAN-Windows-Passwords| \cite{OmgpayloadsPayloadsLibrary}, \verb|-OMG-Credz-Plz| \cite{OmgpayloadsPayloadsLibrary}, or \verb|OMG-AwarenessTraining| \cite{OmgpayloadsPayloadsLibrary} which send the exfiltrated data to a webhook, discord webhook, or Dropbox. Any kind of passing on of data from the O.MG cable to the attacker will require some resource for communication. 


\subsubsection{Initial Access}

Initial Access is about an adversary trying to get access to a target network\cite{MITREATTCK}. One example for how this can be done with keyboard injection are \verb|revshell_windows| and \verb|win_winrm-backdoor| \cite{OmgpayloadsPayloadsLibrary}; payloads that establish remote control over a targeted computer. Through that access point, the network can be infiltrated. Similarly, passwords for networks can be exfiltrated from a target computer \verb|WLAN-Windows-Passwords| \cite{OmgpayloadsPayloadsLibrary}.\\

\subsubsection{Execution}

Execution covers all techniques used to get malicious code running on a target's computer or server using any kind of shell, coding environment, hotkeys, native APIs, or rely on the user to activate code execution, e.g. by clicking on a link or file.\cite{MITREATTCK}\\
Running malicious code is a core component to the way Keyboard Injections are done. It is their alpha and omega and can be found in every script in the collection.

\subsubsection{Persistence} \label{persistence}

Persistence stresses the longevity and robustness of an attack over time, changed credentials. To this end, accounts and access rights can be manipulated, SSH keys stolen or modified, new devices registered for two factor authentication, create or modify system processes (i.e. modifying power shell profile scripts), by using external remote services, or manipulate pre-OS boot mechanisms.\cite{MITREATTCK} \\
An example for this kind of technique is the remote access that can be gained by scripts like \verb|revshell_macOS| \cite{OmgpayloadsPayloadsLibrary} or remote control access establishment as demonstrated by \cite{bojovicRisingThreatHardware2019}. \\
This category also includes attack scheduling, which can easily be achieved by DuckyScript, using the `DELAY` command, the remote trigger, or the Geo fencing \cite{hak5MGCable}.

\subsubsection{Privilege Escalation}

Privilege Escalation includes techniques used to gain higher-level permissions in a network or system, by bypassing account controls, abusing elevation control mechanisms, access token manipulation or theft, account manipulation, braking out of containers to gain access to a host \cite{MITREATTCK}, and many similar techniques as featured in \ref{persistence}. \\
Account manipulation can easily be achieved with the correct recon. Especially if the login of an administrator can be logged \verb|OMGLogger| \cite{OmgpayloadsPayloadsLibrary} or is stored somewhere on the system \verb|SudoSnatch| \cite{OmgpayloadsPayloadsLibrary}, \verb|Everything-Password-Stealer| \cite{OmgpayloadsPayloadsLibrary} privilege escalation is achievable.

\subsubsection{Credential Access}

Credential Access means an adversary tries to steal usernames and passwords \cite{MITREATTCK}, which is a common application for the GitHub scripts, as mentioned in the previous subsections. Examples are \verb|SudoSnatch| \cite{OmgpayloadsPayloadsLibrary}, \verb|Everything-Password-Stealer| \cite{OmgpayloadsPayloadsLibrary}, or with BadUSB;  \cite{muslimImplementationAnalysisUSB2020}. 


\subsubsection{Collection}
After a target has been infiltrated, the data collection process can start. It includes of techniques like Man-in-the-Middle (MIM), compressing data, browser session hijacking, audio and or image capture, clipboard data exfiltration, email collection, keylogging, etc. \cite{MITREATTCK} \\
Keylogging specifically is one of the main features of the O.MG cable, as discussed previously, and further extended by \verb|Persistent_Keylogger-Telegram_Based|  \cite{OmgpayloadsPayloadsLibrary}, image capture is demonstrated by \verb|Screen-Shock| \cite{OmgpayloadsPayloadsLibrary}, or the stealing of fotos by \verb|/ExfiltratePhotosThroughShell| \cite{OmgpayloadsPayloadsLibrary}. 


\subsubsection{Exfiltration}

Exfiltration is about how the stolen data can be relayed to the attacker \cite{MITREATTCK}. As discussed in the section about Data Reconnaissance, exfiltration can happen in a variety of ways. The examples from the GitHub repositories include sending the data to (discord) webhooks, dropbox  ( \verb|ExfiltrateLinuxLogFiles|, \verb|WLAN-Windows-Passwords|, \verb|-OMG-Credz-Plz|, or \verb|OMG-AwarenessTraining|  \cite{OmgpayloadsPayloadsLibrary}).

\subsubsection{Impact}

The techniques belonging to impact are not widely represented in the GitHub repository. They include scripts that try to manipulate, interrupt or destroy systems and data. However, it has been shown by \cite{lawalFacilitatingCyberenabledFraud2022} that it is possible to change, meaning manipulate, data on a target's computer without leaving traces of an attack, thereby framing the computer's user(s) for the data change. \cite{MITREATTCK}


\subsubsection{Intermediate Conclusion}

From the examples above it is apparent, that a wide variety of attacks and techniques are available with DuckyScript and a malicious USB device. Many sections of the \verb|Att&ck| model play a role and are part of various scripts for the O.MG cable that already exist. The section that is most prevalent in the available code base is focused on the gathering and exfiltration of data and gaining access to systems and networks. 


\section{Setup for an HID Injection Attack}

When preparing for an HID injection attack, a malicious actor has to pay attention to the following 5 points;
\begin{enumerate}
    \item Target
    \item Circumstance and Place
    \item Time and Timing
    \item required Hardware
    \item required Software
\end{enumerate}


\subsubsection{Target}

The foremost component to an attack is the target. It determines all other factors. Crucial is the goal of the attack, the level of available access, and the environment and characteristics of the target.\\
The goal determines the type of attack, which will be further specified by the environment, chaarcteristics and access. [where am i going with this, in how far is it even relevant?] 

\section{Implementation} \label{Implementation}

