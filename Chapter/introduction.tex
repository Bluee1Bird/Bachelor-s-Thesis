\chapter{Introduction}

The Protocol of the Universal Serial Bus (USB) does not require any authentication for UBS peripherals. This blind trust can be abused by malicious actors to spoof USB keyboards with devices like UBS sticks, USB dongles, or USB cables. These malicious USB devices communicate to a USB host that they are a keyboard and can then send a series of programmed commands that look like keystrokes to the computer. In this way, access to a host can be gained without any authentication required (provided that the host is unlocked). These attack devices then have full access to a host and can execute any action a human sitting at the computer could do while being much more inconspicuous. 




\section{Motivation}

It looks like the company was infiltrated by a corporate spy, their long-standing, seemingly loyal employee turned on them and executed a series of malicious attacks on the company spanning over multiple days, stealing information and planting incriminating evidence against her employer. Everything points to her, the logs show that the activity came from her account; no traces of malpractice, viruses, or trojans were found on her machine, yet she asserts over and over that she is innocent. \\
No one suspects that this is the doing of a small, innocent-looking USB Cable the employee received for free at a work convention. This cable is a type of BadUSB, although it poses as harmless and cannot be distinguished from any other cable the employee or her bosses have ever seen, it wreaked havoc and started the intricately orchestrated attack as soon as the employee plugged it in. \\
She used the handy cable to replace the damaged and old one connecting her external keyboard to her work computer. As soon as the malicious O.MG cable was connected it sprung into action. The boot script executed and checked the available Wi-Fi networks. Once it concluded that it was at her place of work, and not in her home, like when she first used it, by identifying the target Wi-FI, it sent a signal to the remote Command and Control (C\&C) server that it was in place and ready to. While it waited for further instructions, it diligently, for weeks, transmitted every keystroke signal the employee typed on her external keyboard not only to the laptop but also to the C\&C server. These keystrokes contained credentials, emails, names, notes, and a lot of information about user behaviour. One day, after the employee had left her desk and locked her Windows machine without shutting it off, the cable sprang into action again. From the data it had gathered, it knew about the daily habits of the employee and her break times. Over the span of the next few breaks, it executed a series of attack scripts. Using the logged credentials, it unlocked the computer and started executing the payloads it was sent from its control server on the other side of the planet, inputting up to 890 keystrokes per second, executing commands, extracting information from the laptop, infiltrating the network, placing spyware, and altering information on the host, all while pretending to be the employee working through her break. To avoid detection, it disabled Windows Event Logging and ran processes as background jobs. By the time the employee came back, all that gathered information was sent to the C\&C server and the attackers were able to move on laterally, infiltrating more and more of the company's systems. At the end of their attack series, they destroyed the last remaining indicator of the attack, the cable itself. They used the self-destruct feature to disable not only its malicious capabilities but also its data-transmitting abilities. As a result, the employee thought the cable had broken, it was just a PR gift, after all, and threw it out. With that, she got rid of the only evidence she had to prove that it had not been her executing all those attacks.\\
This framing is the result of a huge effort on the attacker's side, requiring a large amount of information about the target's computer and network. On their end, it was flying blind. The attack could have failed horribly at any step; the employee could have returned earlier from her break and interrupted the execution of one of the attacks, the computer could have made an update at the wrong time, a pop-up could have changed the cursor focus. Although keystroke injection appears to be a straightforward attack, executing it effectively involves substantial operational and technical challenges, particularly in achieving precise timing and accurate execution. Nevertheless, if done correctly and with enough care the effects of such an attack are detrimental. In the best case, the attack vector is never discovered. Any traces the cable leaves on the system will be attributed to the employee, traces on the host computer memory are gone within the first 24 hours, therefore the extent of the attacks cannot be determined. For this to happen, the employee only had to trust a USB cable, and who wouldn't? After all, its only job is to transmit power and data. 



\section{Description of Work}

This thesis aims to shed light on the history of keyboard injection attacks by examining the past of different attack approaches by various actors as well as the responses they triggered and the multitude of defence approaches that were developed. Additionally, it focuses on developing new payloads for the O.MG cable that are not yet publicly listed on the official O.MG GitHub repository by analyzing the MITRE ATT\&CK framework and comparing its subtechniques to those already present. In a second step analyzes USB traffic generated by the O.MG cable to develop a novel detection mechanism and pairs those results with a rate limiter implementation that features two modes. Lastly, it evaluates the developed attacks and defences against each other.

\section{Thesis Outline}
The content of this thesis is split into six chapters. Chapter 2 will introduce the background for this thesis, and give a brief overview of USB and the technical knowledge necessary to understand the subsequent topics. Chapter 3 details the history of USB attacks as they apply to human interface spoofing and injection attacks. Chapters 4 and 5 cover the architecture and implementation of the new payloads and the defence script, detailing their functionalities and specific features. These implementations are evaluated in Chapter 6, which tests the payloads on different devices and ascertains the best configurations for the defence scripts to detect and interrupt O.MG payloads as quickly as possible.

\section{Ethical Considerations}
Attacks as described in this thesis can cause considerable harm to individuals, companies, and communities. For all the reasons explained and examples brought up in section \ref{TheDangersOfUSB} these attacks are not to be taken lightly and the potential for damage is real. For this exact reason, it is important to raise awareness for this kind of attack, research existing vulnerabilities, new developments in the field, and how those can be counteracted. Investigating attacks in particular ones that are not available in a public GitHub repository is an important contribution to the scientific field and outweighs the negative. Just because these payloads cannot easily be found, does not mean they do not already exist. Which arguably makes them even more dangerous to the public and therefore their exploration is even more urgent. \\
In section \ref{defence_history} this thesis explains countermeasures that can be put in place to protect oneself and in section \ref{Defence Methodology} a novel defence against the O.MG will be presented. 
% bring up the black/white /grey hat discussion again? or is his not scientific enough?

